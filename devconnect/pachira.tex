\documentclass[10pt,xcolor=svgnames]{beamer} %Beamer


\usepackage{palatino} %font type
\usepackage{xcolor}
\usepackage[font={small}]{caption}
\usefonttheme{metropolis} %Type of slides
\usefonttheme[onlymath]{serif} %font type Mathematical expressions
\usetheme[progressbar=frametitle,titleformat frame=smallcaps,numbering=counter]{metropolis} %This adds a bar at the beginning of each section.
\useoutertheme[subsection=false]{miniframes} %Circles in the top of each frame, showing the slide of each section you are at

\usepackage{appendixnumberbeamer} %enumerate each slide without counting the appendix

\definecolor{SysBlue}{RGB}{12,120,183}
\definecolor{Progress}{RGB}{56,60,92}
\definecolor{Header}{RGB}{12,120,183}
\definecolor{SlideTitle}{RGB}{0,137,208}
\definecolor{ToC}{RGB}{51,58,66}

\setbeamercolor{progress bar}{fg=Progress} %These are the colours of the progress bar. Notice that the names used are the svgnames
\setbeamercolor{title separator}{fg=SysBlue} %This is the line colour in the title slide
\setbeamercolor{structure}{fg=black} %Colour of the text of structure, numbers, items, blah. Not the big text.
\setbeamercolor{normal text}{fg=black!87} %Colour of normal text
\setbeamercolor{alerted text}{fg=DarkRed!60!Gainsboro} %Color of the alert box
\setbeamercolor{example text}{fg=ToC} %Colour of the Example block text
\setbeamercolor{background canvas}{bg=white}


\setbeamercolor{palette primary}{bg=SlideTitle, fg=white} %These are the colours of the background. Being this the main combination and so one. 
\setbeamercolor{palette secondary}{bg=Header, fg=white}
\setbeamercolor{palette tertiary}{bg=Header, fg=white}
\setbeamercolor{section in toc}{fg=ToC} %Color of the text in the table of contents (toc)

\setbeamertemplate{blocks}[rounded][shadow=true]


%These next packages are the useful for Physics in general, you can add the extras here. 
\usepackage{amsmath,amssymb}
\usepackage{slashed}
\usepackage{cite}
\usepackage{relsize}
\usepackage{caption}
\usepackage{subcaption}
\usepackage{multicol}
\usepackage{booktabs}
\usepackage[scale=2]{ccicons}
\usepackage{pgfplots}
\usepgfplotslibrary{dateplot}
\usepackage{geometry}
\usepackage{xspace}

\newcommand{\themename}{\textbf{\textsc{bluetemp}\xspace}}%metropolis}}\xspace}

\title{Pachira Fund}
\author[Name]{Ian Moore, PhD \inst{$\dagger$}}%With inst, you can change the institution they belong
\subtitle{Can Money Grow on Trees?}

\institute[shortinst]{\inst{$\dagger$} Tokenomics Researcher / Engineer @ Syslabs (email: imoore@syscoin.org) }

\date{November 13, 2023} %Here you can change the date
\titlegraphic{\vspace{-0.5cm}\hfill\includegraphics[scale=0.23]{logo.png}} %You can modify the location of the logo by changing the command \vspace{}. 

\begin{document}
{
\setbeamercolor{background canvas}{bg=white, fg=black}
\setbeamercolor{normal text}{fg=black}
\maketitle
}%This is the colour of the first slide. bg= background and fg=foreground

\metroset{titleformat frame=smallcaps} %This changes the titles for small caps

\begin{frame}{Outline}
  \setbeamertemplate{section in toc}[sections numbered] %This is numbering the sections
  \tableofcontents[hideallsubsections] %You can comment this line if you want to show the subsections in the table of contents
\end{frame}

\section{Introduction}



\begin{frame}{Objectives}

\metroset{block=fill}
\begin{exampleblock}{\textsc{Define our problem}}
\begin{itemize}
  \item Stagnant Pool Problem
  \item Implicitly addressed by Uniswap v3
\end{itemize}  
\end{exampleblock}

\begin{exampleblock}{\textsc{Present our Solution}}
\begin{itemize}
  \item Liquidity Trees
  \item Simulation Results
\end{itemize}
\end{exampleblock}

\begin{exampleblock}{\textsc{Objective Question Revisted}}
\begin{itemize}
  \item Can Money Grow on Trees?
\end{itemize}
\end{exampleblock}

\end{frame}


\section{What are we Solving?}

\begin{frame}{Stagnant Pool Problem} 

\metroset{block=fill}
\begin{exampleblock}{\textsc{Lay Definition:}}
\begin{itemize}
  \item How do we increase $\Delta V$ on stagnated liquidity in pool over some time interval $t$?
\end{itemize}
\end{exampleblock}

\begin{exampleblock}{\textsc{Uniswap v3 has implicitly addressed this problem:}}
\begin{itemize}
  \item Increasing the depth of order book (virtually)
  \item Makes large trades more efficient
  \item Hence increasing the \textit{possibility} of more $\Delta V$
  \item However, does nothing to induce more trading volume
\end{itemize}
\end{exampleblock}


\end{frame}


\begin{frame}{Uniswap v3} 

\begin{figure}[h!]
\includegraphics[width=2.3in]{img/uniswap_v3.png}
\caption{Graphical illustration on how depth is virtually increased using Uniswap v3} 
\label{fig:uniswap_v3}
\end{figure}

\end{frame}

\begin{frame}{Indexing Problem: Posed} 

\metroset{block=fill}
\begin{exampleblock}{\textsc{CPT Indexing Problem}}
\begin{itemize}
  \item Given a position $\Delta L$, what is the indexed value in only one of the two pairing assets (x,y)
\end{itemize}
\end{exampleblock}

\begin{figure}[h!]
\includegraphics[width=2.3in]{img/indexed_tkn_lrg.png}
\label{fig:uniswap_v3}
\end{figure}



\end{frame}


\begin{frame}{Indexing Problem: Solution} 

This system describes the mathematical operations that are effectively being done in the contract code when performing an \textit{efficient} two-step withdrawal (ie, withdrawal both assets + swap one for the remaining) operation: 

\begin{align}
\Delta x &=\frac{\Delta L x}{L}\\
\Delta y &=\frac{\Delta L y}{L}\\
\Delta iy &= \Delta y + \frac{\gamma \Delta x (y - \Delta y)}{(x - \Delta x) + \gamma \Delta x}
\end{align}

\end{frame}


\begin{frame}{Let's Include a new Dimension ...} 
 
\begin{figure}[h!]
\includegraphics[width=3in]{img/simple_tree.png}
\caption{Include a new dimension (ix); liquidity $\Delta L$ gets indexed to $\Delta ix$, and new market is formed on ix-x plane} 
\label{fig:simple_tree}
\end{figure}

\end{frame}

\begin{frame}{How does this Work?} 

\begin{figure}[h!]
\includegraphics[width=3in]{img/compare.png}
\caption{Boxes represent liquidity under CPT curve; creating a new market out of indexed liquidity indexed liquidity is a way to address the stagnant pool problem; in short we've leveraged some of the green, got red and made some extra purple} 
\label{fig:compare}
\end{figure}

\end{frame}



\section{Liquidity Trees}

\begin{frame}{Full Tree} 

\begin{figure}[h!]
\includegraphics[width=3in]{img/full_tree.png}
\caption{Full CPT liquidity tree represented as a computational tree structure comprised of left-sided, right-sided and synthetic pools } 
\label{fig:full_tree}
\end{figure}

Liquidity Trees can be represented as computational graphs where nodes are denoted as DEX operations and arcs are denoted as indexed capital transitioning between the parent node and the child

\end{frame}

\begin{frame}{Sub Trees} 

\begin{figure}[h!]
\includegraphics[width=3in]{img/sub_trees.png}
\caption{Sub-trees comprised of: (TOP LEFT) left tree; (TOP CENTER) right tree; (TOP RIGHT) double tree; (BOTTOM LEFT) equivalent tree; (BOTTOM CENTER) opposing Tree} 
\label{fig:sub_trees}
\end{figure}

\end{frame}


\section{Simulation Results}

\begin{frame}{Left Tree}

\begin{figure}[h!]
\includegraphics[width=3in]{img/left_tree.png}
\label{fig:left_tree}
\end{figure}

\end{frame}


\begin{frame}{Left Tree: Simulated Price / Revenue (1)}

\begin{figure}[h!]
\includegraphics[width=3in]{img/simulation.jpg}
\label{fig:simulation}
\end{figure}

\end{frame}


\begin{frame}{Left Tree: Simulated Price / Revenue (2)}


\begin{table}[h]
\tiny
\centering
\begin{tabular}{ |l|c|c|c| } 
\hline
 \textbf{Metric} & \multicolumn{3}{|c|}{\textbf{Totals}}\\
\hline 
 Revenue (LP) / LP Liquidity & \multicolumn{3}{|c|}{19.4\% }\\
 Revenue (LP+LP1+LP2) / LP Liquidity & \multicolumn{3}{|c|}{26.3\% }\\
 Revenue Boost (Indexed Liquidity)  & \multicolumn{3}{|c|}{35.61\%}\\ 
 Percentage Indexed & \multicolumn{3}{|c|}{7.51\%}\\ 
 \hline
\hline
	& \multicolumn{3}{|c|}{\textbf{Sub-totals}} \\
  & \textbf{LP (ETH-DAI)} & \textbf{LP1 (iETH-ETH)} & \textbf{LP2 (iETH-DAI)} \\
\hline
Liquidity  & \$1,559,145 & \$71,961 & \$162,118\\
Revenue  & \$302,140 & \$57,927 & \$49,673\\
Revenue / Liquidity  & 19.4\% & 80.5\% & 30.64\%\\ 
\hline
\end{tabular}
\caption{Metrics harvested from Left-tree simulation using ETH \& DAI (Jan 2018 to Oct 2023)}
\label{table:simulator_components}
\end{table}

\end{frame}



\section{Our Objective Question}

\begin{frame}{In Summary}


\metroset{block=fill}
\begin{exampleblock}{\textsc{Stagnant Pool Problem}}
\begin{itemize}
  \item Defined stagnant pool problem and how it can be addressed
\end{itemize}
\end{exampleblock}

\begin{exampleblock}{\textsc{Liquidity Trees: A new DeFi Primitive}}
\begin{itemize}
  \item New DeFi primitive, which we call Liquidity Trees 
  \item Utilizes fully collateralized liquidity, and can be represented as undirected graphs or algebraically in $\mathbb{R}^n$
  \item Simulations support our reasoning
\end{itemize}
\end{exampleblock}

\begin{exampleblock}{\textsc{Pachira Token Launch}}
\begin{itemize}
  \item Combining Liquidity Trees with a governance system for Pachira Fund token launch
\end{itemize}
\end{exampleblock}



\end{frame}


\begin{frame}[standout]
  Thank you! 
\end{frame}

\begin{frame}{Appendix 1: Pachira Token Launch}
\begin{figure}[h!]
\includegraphics[width=3in]{img/dex_forest_single_tree.png}
\label{fig:dex_forest}
\end{figure}
\end{frame}

\begin{frame}{Appendix 2: Reserve Swap Prices}
\begin{figure}[h!]
\includegraphics[width=3.5in]{img/single_swap_prices.png}
\label{fig:swap_prices}
\end{figure}
\end{frame}

\begin{frame}{Appendix 3: Pachira Token Launch}
\begin{figure}[h!]
\includegraphics[width=4.2in]{img/dex_forest.png}
\label{fig:dex_forest}
\end{figure}
\end{frame}

\begin{frame}{Appendix 4: Reserve Swap Prices}
\begin{figure}[h!]
\includegraphics[width=3.5in]{img/swap_prices.jpg}
\label{fig:swap_prices}
\end{figure}
\end{frame}

\end{document}